\documentclass[a4paper,10pt,openright]{report}
% \usepackage{a4wide}
\usepackage{amsmath,amsfonts}
\usepackage{amsthm}
\usepackage{amssymb}
\usepackage[italian]{babel}
\usepackage{amsfonts}
\usepackage[T1]{fontenc}
\usepackage[utf8x]{inputenc}
\usepackage{enumerate}
\usepackage{verbatim}
\usepackage{graphicx}
\usepackage{verbatim}
\usepackage{faktor}
\usepackage{bbm}
\usepackage{tocbibind}


\usepackage[all]{xy}
% \usepackage[sc]{mathpazo}
\usepackage{textcomp}

\SetUnicodeOption{mathletters}
\SetUnicodeOption{autogenerated}


\theoremstyle{plain}
\newtheorem{teo}{Teorema}[section]
\newtheorem{lem}[teo]{Lemma}
\newtheorem{prop}[teo]{Proposizione}
\newtheorem{cor}[teo]{Corollario}
\newtheorem*{teo*}{Teorema}
\newtheorem{ipotesi}{ipotesi}

\theoremstyle{definition}
\newtheorem{dfn}{Definizione}[section]
\newtheorem*{dfn*}{Definizione}
\newtheorem{ex}[teo]{Esempio}
\newtheorem{oss}[teo]{Osservazione}

\usepackage{amsmath}
\usepackage{caption}

\renewcommand{\t}{\texttt}
\renewcommand{\o}{\circ}
\newcommand{\wt}{\widetilde}
\newcommand{\tri}{\triangle}
\newcommand{\R}{\mathbb{R}}
\newcommand{\Z}{\mathbb{Z}}
\newcommand{\N}{\mathbb{N}}
\newcommand{\Hom}{\text{Hom}}
\newcommand{\Ker}{\text{Ker}}
\newcommand{\im}{\text{Im}}
\newcommand{\s}{\sigma}
%\newcommand{\}{\mathbb{R}}
\newcommand{\Ra}{\Rightarrow}
\newcommand{\ra}{\rightarrow}
\newcommand{\cu}{\subseteq}
\newcommand{\scal}[2]{\left\langle \, #1 \, , #2 \, \right\rangle}
\newcommand{\norm}[1]{\left\lVert#1\right\rVert}
\newcommand*\di{\mathop{}\!\mathrm{d}}
%opening
\title{Il Teorema della Sfera in Geometria Riemanniana}
\author{}

\begin{document}
\begin{center}
 \textbf{\large{Il Continuum Random Tree}}\\
\vspace{.3cm}
Candidato: Lorenzo Beretta\\
Relatore: Prof. Franco Flandoli\\
\end{center}
\vspace{.5cm}

Lo scopo di questa trattazione è di illustrare la costruzione dell'albero continuo aleatorio (CRT) svolta da Aldous nella triade di articoli \cite{Ald1},\cite{Ald2} e \cite{Ald3} per poi approfondire un importante risultato che questo strumento permette di formalizzare: la convergenza sotto opportune ipotesi degli alberi Galton-Watson al \textit{brownian continuum random tree}.\\
\\
Nella letteratura matematica votata alle applicazioni si trovano molti modelli di alberi aleatori per i quali sono state ampiamente studiate le distribuzioni asintotiche di alcuni parametri (altezza, profilo, distanza media tra i nodi, etc...) utilizzando strumenti combinatorici.
Il continuum random tree sarà l'oggetto astratto che otterremo come limite di scala di certe famiglie di alberi aleatori, fornendoci un'altro strumento per studiare tali distribuzioni. In particolare vedremo che la sua versione browniana sarà il limite di un modello molto utilizzato di albero aleatorio: il Galton-Watson tree, ovvero l'albero genealogico.\\
\\
In questa tesi in primo luogo rivisiteremo un risultato classico della teoria della convergenza di processi: il principio di invarianza di Donsker, risultato che fa da precursore a quelli di Aldous e condivide con essi il metodo: dimostrare la convergenza di processi mostrando prima quella dei marginali finito-dimensionali per concludere sfruttando la \textit{tightness}.\\
Costruiremo poi il continuum random tree come un oggetto astratto avente due diverse rappresentazioni: la prima attraverso i sottoalberi aleatori finiti ottenuti con un campionamento aleatorio dei suoi vertici e la seconda attraverso la corrispondenza biunivoca con particolari funzioni continue che costituiscono la naturale estensione della nozione di \textit{funzione contorno} per un albero, ovvero l'interpolazione lineare della funzione ottenuta dalla DFS (\textit{depth first search}).\\
\\
Il lavoro teorico per giungere a questa duplice caratterizzazione del CRT sarà poi ripagato dalla dimostrazione della convergenza delle funzioni contorno dei \textit{Galton-Watson trees} aventi distribuzione subcritica e varianza finita all'escursione browniana, risultato che ci permette di esprimere risultati asintotici sulle distribuzioni di funzioni della "forma" di questi alberi come funzionali dell'escursione browniana, oggetto ben noto in letteratura.

\begin{thebibliography}{99} 

\bibitem{Ald1} Aldous, D. J. \textit{The continuum random tree. I.} 1991, Ann. Probab. 19, 1-28.

\bibitem{Ald2} Aldous, D. J. \textit{The continuum random tree II: an overview.} 1991, Proc. Durham Symp. Stochastic Analysis 1990, 23-70. Cambridge Univ. Press.

\bibitem{Ald3} Aldous, D. J.  \textit{The Continuum Random Tree III.} 1993, Ann. Probab. 21, 248--289.



\end{thebibliography}

\end{document}
